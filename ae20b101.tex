\documentclass[a4paper, 12pt]{article}
\usepackage[left=2cm, right=2cm, top=2cm, bottom=2cm]{geometry}
\begin{document}
\title{Assignment 4}
\author{Abhijeet Rajendran Nair AE20B101}
\date{\today}
\maketitle

The following are the Maxwell's equations:
\begin{equation}
\nabla \cdot \mathbf{E} = \frac{\rho}{\epsilon_0}
\end{equation}

\begin{equation}
\nabla \cdot \mathbf{B} = 0
\end{equation}

\begin{equation}
\nabla \times \mathbf{E} = - \frac{\partial \mathbf{B}}{\partial t}
\end{equation}

\begin{equation}
\nabla \times \mathbf{B} = \mu_0 (\mathbf{J} + \epsilon_0 \frac{\partial \mathbf{E}}{\partial t})
\end{equation}

In these equations, the terms used are
\begin{itemize}
  \item $\mathbf{E}$ denotes electric field at a point,
  \item $\rho$ denotes volume charge density,
  \item $\epsilon_0$ denotes permeability of free space,
  \item $\mathbf{B}$ denotes magnetic field at a point,
  \item $\mu_0$ denotes permittivity of free space,
  \item $\mathbf{J}$ denotes current density
\end{itemize}

These four equations describe how electric and magnetic fields behave 
and are formed by charges, currents and changes 
in the fields. Together with the Lorentz force law, 
they form the foundation of classical electromagneticsm, electric circuits 
and optics.

\end{document}
